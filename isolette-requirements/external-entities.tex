\chapter{External Entities}
\label{chap:external-entities}

The following sections describe the external entities with which the Thermostat directly
interacts: the Temperature Sensor, the Operator Interface, and the Heat Source. The monitored
and controlled variables associated with each entity are listed, along with any environmental
assumptions made about the entity. An Isolette external entity is also defined to specify
environmental assumptions that span more than one external entity.

\section{Isolette}
\label{sec:isolette}

An Isolette is an incubator for an Infant that provides controlled temperature, humidity, and
oxygen (if necessary). It encompasses the Thermostat, the Temperature Sensor, the Operator
Interface, and the Heat Source. The following environmental assumptions are made by the
Thermostat about the Isolette

\begin{itemize}
\item EA-IS-1: When the Heat Source is turned on and the Isolette is properly shut, the
      Current Temperature will increase at a rate of no more than 1°F per minute.

      Rationale: If the Current Temperature can increase at a rate of more than 1°F per minute,
      the Thermostat may not be able to turn the Heat Source off quickly enough to maintain
      the Desired Temperature Range unless the allowed latency specified for the Heat Control
      is reduced.
\item EA-IS-2: When the Heat Source is turned off and the Isolette is properly shut, the
      Current Temperature will decrease at a rate of no more than 1°F per minute.

      Rationale: If the Current Temperature can decrease at a rate of more than 1°F per
      minute, the Thermostat may not be able to turn the Heat Source on quickly enough to
      maintain the Desired Temperature Range unless the allowed latency specified for the
      Heat Control is reduced.
\end{itemize}

\section{Temperature Sensor}
\label{sec:temperature-sensor}

The Temperature Sensor provides the Current Temperature of the Air in the Isolette to the
Thermostat. The monitored variables are shown in table ~\ref{tab:temp-sensor}.

\begin{table}
\begin{tabular}{|l|l|l|l|l|}
\hline
Name & Type & Range & Units & Physical Interpretation \\\hline
Current Temperature & Real & [68.0..105.0] & °F & Current air temperature inside Isolette \\\cline{2-4}
     & Current & \textbullet Invalid, Valid &  &  \\\hline
\end{tabular}
\caption{Thermostat Monitored Variables for Temperature Sensor}
\label{tab:temp-sensor}
\end{table}

\begin{itemize}
\item Table ~\ref{tab:temp-sensor} denotes initial value
\end{itemize}

The following environmental assumptions are made:

\begin{itemize}
\item EA-TS-1: The Current Temperature will be provided to the Thermostat in degrees
      Fahrenheit

      Rationale: Consistency with environmental-assumption Operator Interface EA-OI-1
\item EA-TS-2: The Current Temperature will be sensed to an accuracy of ±0.1°F.

      Rationale: An accuracy of 0.1°F is necessary to ensure the Thermostat can turn the Heat
      Source on and off quickly enough to maintain the Desired Temperature Range.
\item EA-TS-3: The Current Temperature will cover the range of at least 68.0° to 105.0°F.

      Rationale: This is the specified range of operation of the Isolette. The lower end of this
      range is useful for monitoring an Isolette that is warming to the Desired Temperature
      Range. The upper end is greater than the Upper Alarm Temperature to ensure that the
      Current Temperature will be sensed across the entire Alarm Temperature Range.
\end{itemize}

\section{Heat Source}
\label{sec:heat-source}

The Heat Source heats the Air in the Isolette. It is turned on and off by changing the value of the
Heat Control controlled variable. The controlled variables are shown in table ~\ref{tab:heat-source-variables}. No
environmental assumptions are made.

\begin{table}
\begin{tabular}{|l|l|l|l|l|}
\hline
Name & Type & Range & Units & Physical Interpretation \\\hline
Heat Control & Enumerated & Off, On &  & Command to turn Heat Source on and off \\\hline
\end{tabular}
\caption{Thermostat Controlled Variables for Heat Source}
\label{tab:heat-source-variables}
\end{table}

\section{Operator Interface}
\label{sec:operator-interface}

The Operator Interface provides the Operator Settings for the Thermostat and receives Operator
Feedback from the Thermostat. The environmental assumptions associated with the Operator
Interface are quite strong, which simplifies the manage Operator Interface Function. If these
assumptions were not satisfied by the Operator Interface external entity, the Manage Operator
Interface Function would need to be strengthened to ensure consistent inputs to the Thermostat.
The monitored and controlled variables are shown in tables ~\ref{tab:operator-interface-variables} and ~\ref{tab:oi-controlled-variables}, respectively.

\begin{table}
\begin{tabular}{|l|l|l|l|l|}
\hline
Name & Type & Range & Units & Physical Interpretation \\\hline
Operator Settings &  &  &  & Thermostat settings provided by operator \\\hline
\multicolumn{4}{|l|}{Desired Temperature Range} & Desired range of Isolette temperature \\\hline
Lower Desired & Integer & [97..99] & °F & Lower value of Desired Temperature Range \\\cline{2-4}
Temperature & Status & \textbullet Invalid, Valid &  & \\\hline
Upper Desired & Integer & [98..100] & °F & Upper value of Desired Temperature Range \\\cline{2-4}
Temperature & Status & \textbullet Invalid, Valid & & \\\hline
\multicolumn{4}{|l|}{Alarm Temperature Range} & Active Alarm when outside of this range \\\hline
Lower Alarm & Integer & [93..98] & °F & Lower value of Alarm Temperature Range \\\cline{2-4}
Temperature & Status & \textbullet Invalid, Valid &  & \\\hline
Upper Alarm & Integer & [98..100] & °F & Upper value of Alarm Temperature Range \\\cline{2-4}
Temperature & Status & \textbullet Invalid, Valid & & \\\hline
\end{tabular}
\caption{Thermostat Monitored Variables for Operator Interface}
\label{tab:operator-interface-variables}
\end{table}

\begin{table}
\begin{tabular}{|l|l|l|l|l|}
\hline
Name & Type & Range & Units & Physical Interpretation \\\hline
Operator Feedback &  &  &  & Information provided back to the operator \\\hline
Regulator Status & Enumerated & Init, On, Failed &  & Status of the Thermostat Regulator Function \\\hline
Monitor Status & Enumerated & Init, On, Failed &  & Status of the Thermostat Monitor Function \\\hline
Display Temperature & Integer & [68..105] & °F & Displayed temperature of Isolette \\\hline
Alarm & Enumerated & Off, On &  & Command to turn Alarm on or off \\\hline
\end{tabular}
\caption{Thermostat Controlled Variables for Operator Interface}
\label{tab:oi-controlled-variables}
\end{table}

\begin{itemize}
\item Table ~\ref{tab:operator-interface-variables} denotes initial value
\end{itemize}

The following environmental assumptions are made:

\begin{itemize}
\item EA-OI-1: All temperatures will be entered and displayed in degrees Fahrenheit.

      Rationale: Minimize the complexity of this example. An actual system would probably
      support Celsius or perhaps both Fahrenheit and Celsius
\item EA4-OI-2: All temperatures will be set and displayed by the operators in increments of
      1°F.

      Rationale: Marketing studies have shown that customers prefer to set temperatures in
      1 degree increments. A resolution 1°F is sufficient to be consistent with the functional
      and performance requirements specified in the rest of the document.
\item EA-OI-3: The Lower Alarm Temperature will always be $\geq$93°F.

      Rationale: Exposure to temperatures less than 93°F will result in hypothermia, which can
      lead to death within a few minutes for severely ill preterm infants.
\item EA-OI-4: The Lower Alarm Temperature will always be less than or equal to the Lower
      Desired Temperature of -1°F.

      Rationale: If the Lower Alarm Temperature is greater than or equal to the Lower Desired
      Temperature, the Alarm could be activated while the Current Temperature is still in the
      Desired Temperature Range.
\item EA-OI-5: The Lower Desired Temperature will always be $\geq$97°F.
      Rationale: Exposing the Infant to temperatures lower than 97°F may result in excessive
      heat loss and drop in heart rate secondary to metabolic acidosis.
\item EA-OI-6: The Lower Desired Temperature will always be less than or equal to the Upper
      Desired Temperature of -1°F.

      Rationale: If the Lower Desired Temperature is greater than or equal to the Upper
      Desired Temperature, it is unclear if the Heat Source should be on or off. This may result
      in excessive cycling of the Heat Source.
\item EA-OI-7: The Upper Desired Temperature will always be $\leq$100°F.
      Rationale: Exposing the Infant to temperatures greater than 100°F may result in an
      incorrect diagnosis of fever resulting in aggressive evaluation (blood culture and lumbar
      puncture) and treatment for infection.
\item EA-OI-8: The Upper Alarm Temperature will always be greater than or equal to the
      Upper Desired Temperature of 1°F.

      Rationale: If the Upper Alarm Temperature is less than or equal to the Upper Desired
      Temperature, the Alarm could be activated while the Current Temperature is still in the
      Desired Temperature Range.
\item EA-OI-9: The Upper Alarm Temperature will always be $\leq$103°F.

      Rationale: Exposure to temperatures greater than 103°F will result in hyperthermia,
      which can lead to cardiac arrhythmias and febrile seizures within a few minutes.
\item EA-OI-9: The Display Temperature will cover the range of at least 68.0° to 105.0°F.

      Rationale: This is the specified range of operation of the Isolette. The lower end of this
      range is useful for monitoring an Isolette that is warming to the Desired Temperature
      Range. The upper end is set to be greater than the maximum Upper Alarm Temperature.
\end{itemize}